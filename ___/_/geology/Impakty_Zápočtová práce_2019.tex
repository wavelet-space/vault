% Options for packages loaded elsewhere
\PassOptionsToPackage{unicode}{hyperref}
\PassOptionsToPackage{hyphens}{url}
%
\documentclass[
]{article}
\usepackage{lmodern}
\usepackage{amssymb,amsmath}
\usepackage{ifxetex,ifluatex}
\ifnum 0\ifxetex 1\fi\ifluatex 1\fi=0 % if pdftex
  \usepackage[T1]{fontenc}
  \usepackage[utf8]{inputenc}
  \usepackage{textcomp} % provide euro and other symbols
\else % if luatex or xetex
  \usepackage{unicode-math}
  \defaultfontfeatures{Scale=MatchLowercase}
  \defaultfontfeatures[\rmfamily]{Ligatures=TeX,Scale=1}
\fi
% Use upquote if available, for straight quotes in verbatim environments
\IfFileExists{upquote.sty}{\usepackage{upquote}}{}
\IfFileExists{microtype.sty}{% use microtype if available
  \usepackage[]{microtype}
  \UseMicrotypeSet[protrusion]{basicmath} % disable protrusion for tt fonts
}{}
\makeatletter
\@ifundefined{KOMAClassName}{% if non-KOMA class
  \IfFileExists{parskip.sty}{%
    \usepackage{parskip}
  }{% else
    \setlength{\parindent}{0pt}
    \setlength{\parskip}{6pt plus 2pt minus 1pt}}
}{% if KOMA class
  \KOMAoptions{parskip=half}}
\makeatother
\usepackage{xcolor}
\IfFileExists{xurl.sty}{\usepackage{xurl}}{} % add URL line breaks if available
\IfFileExists{bookmark.sty}{\usepackage{bookmark}}{\usepackage{hyperref}}
\hypersetup{
  hidelinks,
  pdfcreator={LaTeX via pandoc}}
\urlstyle{same} % disable monospaced font for URLs
\usepackage{longtable,booktabs}
% Correct order of tables after \paragraph or \subparagraph
\usepackage{etoolbox}
\makeatletter
\patchcmd\longtable{\par}{\if@noskipsec\mbox{}\fi\par}{}{}
\makeatother
% Allow footnotes in longtable head/foot
\IfFileExists{footnotehyper.sty}{\usepackage{footnotehyper}}{\usepackage{footnote}}
\makesavenoteenv{longtable}
\setlength{\emergencystretch}{3em} % prevent overfull lines
\providecommand{\tightlist}{%
  \setlength{\itemsep}{0pt}\setlength{\parskip}{0pt}}
\setcounter{secnumdepth}{-\maxdimen} % remove section numbering

\date{}

\begin{document}

\hypertarget{simulace-dopadu-mimozemskuxe9ho-tux11blesa}{%
\section{Simulace dopadu mimozemského
tělesa}\label{simulace-dopadu-mimozemskuxe9ho-tux11blesa}}

Závěrečná práce pro předmět \emph{Impaktové kráterování a šoková
metamorfóza} -- David Landa (2019).

\hypertarget{uxfavod}{%
\subsection{Úvod}\label{uxfavod}}

Pro simulaci dopadu mimozemského tělesa s níže uvedenými počátečními
hodnotami jsem použil program dostupný na
https://impact.ese.ic.ac.uk/ImpactEarth/ImpactEffects.

\textbf{Počáteční hodnoty}

\begin{itemize}
\tightlist
\item
  průměr 1500 m
\item
  hustotě 3 g/cm³
\item
  rychlosti dopadu 17 km/s
\item
  pod úhlem 45°
\item
  dopad do sedimetárního terčového materiálu
\item
  dopad ve vzdálenosti 10, 50, 100 km
\end{itemize}

Jako místo dopadu jsem vybral Paříž o přibližné rozloze 100 km². V
dalším textu uvedu jednotlivé charakteristiky dopadu tělesa a účinky na
infrastrukturu. Údaje pro jednotlivé vzdálenosti čísluji v tabulkách
jednoduše jako 1 (10 km), 2 (50 km) a 3 (100 km). Úplný výstup programu
s popisem jednotlivých vlastností dopadu mimozemského tělesa je uveden v
příloze.

\hypertarget{porovnuxe1nuxed-charakteristik-dopadu-pro-jednotlivuxe9-vzduxe1lenosti}{%
\subsection{Porovnání charakteristik dopadu pro jednotlivé
vzdálenosti}\label{porovnuxe1nuxed-charakteristik-dopadu-pro-jednotlivuxe9-vzduxe1lenosti}}

\hypertarget{rozsah-kruxe1teru}{%
\subsubsection{Rozsah kráteru}\label{rozsah-kruxe1teru}}

\begin{itemize}
\tightlist
\item
  Transient Crater Diameter: 14.5 km
\item
  Transient Crater Depth: 5.12 km
\item
  Final Crater Diameter: 20.6 km
\item
  Final Crater Depth: 736 meters - The crater formed is a complex
  crater.
\item
  The volume of the target melted or vaporized is 4.78 km³
\item
  Roughly half the melt remains in the crater, where its average
  thickness is 29 meters.
\end{itemize}

Vysvětlení uvedených pojmů lze nalézt např. na
https://www.purdue.edu/impactearth/Home/Glossary

\hypertarget{vyzuxe1ux159enuxe9-teplo}{%
\subsubsection{Vyzářené teplo}\label{vyzuxe1ux159enuxe9-teplo}}

Vyzářené teplo, neboli termální radiace, popisuje parametry zářící
sféry, konkrétně její absolutní poloměr, relativní velikost vůčí slunci,
tak jak ho vidíme ze Země, zaznamenané teplo na m², dobu záření a
tepelný tok opět relativní ke slunci.

\begin{longtable}[]{@{}llllll@{}}
\toprule
\begin{minipage}[b]{0.02\columnwidth}\raggedright
\#\strut
\end{minipage} & \begin{minipage}[b]{0.21\columnwidth}\raggedright
Visible fireball radius (km)\strut
\end{minipage} & \begin{minipage}[b]{0.13\columnwidth}\raggedright
Size (relative to the sun)\strut
\end{minipage} & \begin{minipage}[b]{0.14\columnwidth}\raggedright
Thermal Exposure (Joules/m²)\strut
\end{minipage} & \begin{minipage}[b]{0.17\columnwidth}\raggedright
Duration of Irradiation (minutes)\strut
\end{minipage} & \begin{minipage}[b]{0.17\columnwidth}\raggedright
Radiant flux (relative to the sun)\strut
\end{minipage}\tabularnewline
\midrule
\endhead
\begin{minipage}[t]{0.02\columnwidth}\raggedright
1\strut
\end{minipage} & \begin{minipage}[t]{0.21\columnwidth}\raggedright
Your position is inside the fireball\strut
\end{minipage} & \begin{minipage}[t]{0.13\columnwidth}\raggedright
415\strut
\end{minipage} & \begin{minipage}[t]{0.14\columnwidth}\raggedright
3.63 x 109\strut
\end{minipage} & \begin{minipage}[t]{0.17\columnwidth}\raggedright
3.95\strut
\end{minipage} & \begin{minipage}[t]{0.17\columnwidth}\raggedright
15300\strut
\end{minipage}\tabularnewline
\begin{minipage}[t]{0.02\columnwidth}\raggedright
2\strut
\end{minipage} & \begin{minipage}[t]{0.21\columnwidth}\raggedright
18.1\strut
\end{minipage} & \begin{minipage}[t]{0.13\columnwidth}\raggedright
82.1\strut
\end{minipage} & \begin{minipage}[t]{0.14\columnwidth}\raggedright
1.43 x 108\strut
\end{minipage} & \begin{minipage}[t]{0.17\columnwidth}\raggedright
3.95\strut
\end{minipage} & \begin{minipage}[t]{0.17\columnwidth}\raggedright
604\strut
\end{minipage}\tabularnewline
\begin{minipage}[t]{0.02\columnwidth}\raggedright
3\strut
\end{minipage} & \begin{minipage}[t]{0.21\columnwidth}\raggedright
17.5\strut
\end{minipage} & \begin{minipage}[t]{0.13\columnwidth}\raggedright
39.7\strut
\end{minipage} & \begin{minipage}[t]{0.14\columnwidth}\raggedright
3.43 x 107\strut
\end{minipage} & \begin{minipage}[t]{0.17\columnwidth}\raggedright
3.95\strut
\end{minipage} & \begin{minipage}[t]{0.17\columnwidth}\raggedright
145\strut
\end{minipage}\tabularnewline
\bottomrule
\end{longtable}

Tepelné účinky dopadu jsou pro všechny zdálenosti stejné, konkrétně:

\begin{itemize}
\tightlist
\item
  Vznícení papíru
\item
  Vznícení oděvů
\item
  Vznícení překližky
\item
  Vznícení listantých stromů
\item
  Vznícení travin
\item
  Popáleniny 3. stupně
\end{itemize}

\hypertarget{seismickuxe9-efekty}{%
\subsubsection{Seismické efekty}\label{seismickuxe9-efekty}}

\begin{longtable}[]{@{}lll@{}}
\toprule
\# & Arrival Time of Shaking (s) & Magnitude\tabularnewline
\midrule
\endhead
1 & 2 & 8.1\tabularnewline
2 & 10 & 8.1\tabularnewline
3 & 20 & 8.1\tabularnewline
\bottomrule
\end{longtable}

Podle Mercalliho škály je pro vzdálenost 10 a 50 km poškození shodné
(stupěň X., XI.) a to kompletní poničení nadzemní i podzemní
infrastruktury. Pro vzdálenost 100 km již většina budov neutrpí nevratné
poškozeni, popadají však komíny, zdi (stupeň VII., VIII.).

\hypertarget{vyvrux17eeniny}{%
\subsubsection{Vyvrženiny}\label{vyvrux17eeniny}}

V případě vdálenosti 10 km jsem přímo v centru kráteru, proto žádná data
o vyvrženinách nejsou dostupná.

\begin{longtable}[]{@{}lllll@{}}
\toprule
\begin{minipage}[b]{0.02\columnwidth}\raggedright
\#\strut
\end{minipage} & \begin{minipage}[b]{0.17\columnwidth}\raggedright
Arrival Time (minutes)\strut
\end{minipage} & \begin{minipage}[b]{0.31\columnwidth}\raggedright
Ejecta Type\strut
\end{minipage} & \begin{minipage}[b]{0.18\columnwidth}\raggedright
Average Ejecta Thickness\strut
\end{minipage} & \begin{minipage}[b]{0.17\columnwidth}\raggedright
Mean Fragment Diameter\strut
\end{minipage}\tabularnewline
\midrule
\endhead
\begin{minipage}[t]{0.02\columnwidth}\raggedright
2\strut
\end{minipage} & \begin{minipage}[t]{0.17\columnwidth}\raggedright
1.69\strut
\end{minipage} & \begin{minipage}[t]{0.31\columnwidth}\raggedright
jemný prach s občasnými velkými fragmenty\strut
\end{minipage} & \begin{minipage}[t]{0.18\columnwidth}\raggedright
3.15 meters\strut
\end{minipage} & \begin{minipage}[t]{0.17\columnwidth}\raggedright
83.5 cm\strut
\end{minipage}\tabularnewline
\begin{minipage}[t]{0.02\columnwidth}\raggedright
3\strut
\end{minipage} & \begin{minipage}[t]{0.17\columnwidth}\raggedright
2.4\strut
\end{minipage} & \begin{minipage}[t]{0.31\columnwidth}\raggedright
jemný prach s občasnými velkými fragmenty\strut
\end{minipage} & \begin{minipage}[t]{0.18\columnwidth}\raggedright
39.4 cm\strut
\end{minipage} & \begin{minipage}[t]{0.17\columnwidth}\raggedright
13.3 cm\strut
\end{minipage}\tabularnewline
\bottomrule
\end{longtable}

\hypertarget{ruxe1zovuxe1-vlna}{%
\subsubsection{Rázová vlna}\label{ruxe1zovuxe1-vlna}}

Hodnoty charakteristik příchozí rázové neboli tlakové vzduchové vlny se
vzrůstající vzdáleností klesají. Z toho samozřejmě vyplývá, že ničivé
následky rázové vlny se vzrůstající vzdáleností klesají. U posledních
dvou vzdáleností je uvedeno, že zvuk může způsobit již jen bolest.

\begin{longtable}[]{@{}lllll@{}}
\toprule
\begin{minipage}[b]{0.03\columnwidth}\raggedright
\#\strut
\end{minipage} & \begin{minipage}[b]{0.12\columnwidth}\raggedright
Arrival Time\strut
\end{minipage} & \begin{minipage}[b]{0.29\columnwidth}\raggedright
Peak Overpressure\strut
\end{minipage} & \begin{minipage}[b]{0.22\columnwidth}\raggedright
Max Wind Velocity (m/s)\strut
\end{minipage} & \begin{minipage}[b]{0.21\columnwidth}\raggedright
Sound Intensity (dB)\strut
\end{minipage}\tabularnewline
\midrule
\endhead
\begin{minipage}[t]{0.03\columnwidth}\raggedright
1\strut
\end{minipage} & \begin{minipage}[t]{0.12\columnwidth}\raggedright
30.3 seconds\strut
\end{minipage} & \begin{minipage}[t]{0.29\columnwidth}\raggedright
3.54e+07 Pa = 354 bars\strut
\end{minipage} & \begin{minipage}[t]{0.22\columnwidth}\raggedright
4780\strut
\end{minipage} & \begin{minipage}[t]{0.21\columnwidth}\raggedright
151 (Dangerously Loud)\strut
\end{minipage}\tabularnewline
\begin{minipage}[t]{0.03\columnwidth}\raggedright
2\strut
\end{minipage} & \begin{minipage}[t]{0.12\columnwidth}\raggedright
2.53 minutes\strut
\end{minipage} & \begin{minipage}[t]{0.29\columnwidth}\raggedright
928000 Pa = 9.28 bars\strut
\end{minipage} & \begin{minipage}[t]{0.22\columnwidth}\raggedright
731\strut
\end{minipage} & \begin{minipage}[t]{0.21\columnwidth}\raggedright
119 (May cause ear pain)\strut
\end{minipage}\tabularnewline
\begin{minipage}[t]{0.03\columnwidth}\raggedright
3\strut
\end{minipage} & \begin{minipage}[t]{0.12\columnwidth}\raggedright
5.05 minutes\strut
\end{minipage} & \begin{minipage}[t]{0.29\columnwidth}\raggedright
207000 Pa = 2.07 bars\strut
\end{minipage} & \begin{minipage}[t]{0.22\columnwidth}\raggedright
293\strut
\end{minipage} & \begin{minipage}[t]{0.21\columnwidth}\raggedright
106 (May cause ear pain)\strut
\end{minipage}\tabularnewline
\bottomrule
\end{longtable}

\hypertarget{zuxe1vux11br}{%
\subsection{Závěr}\label{zuxe1vux11br}}

Paříž leží při udávané rozloze zcela v zóně kráteru o výsledném průměru
20 km. Předpokládáme tedy kompletní zničení města. Podle předpokladů se
vzrůstající vzdáleností klesá většina hodnot a tedy i ničivých účinků
impaktu. Zajímavostí může být, že sesmické efekty si drží konstantní
hodnoty (magnitudo) i na velmi velkou vzdálenost a navíc mohou vyvolat i
sekundární ničivé události jako např. tsunami.

\hypertarget{dodatek}{%
\subsection{Dodatek}\label{dodatek}}

Výstup programu pro jednotlivé vzdálenosti je uveden v příloze.

\end{document}
