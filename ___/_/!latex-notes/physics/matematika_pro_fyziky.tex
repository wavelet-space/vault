\part{Přehled Matematiky pro fyziky}

\chapter{Matematika pro fyziky I}

Obsah přednášky odpovídající předmětu Matematika pro fyziky I.

\section{Úvod do variačního počtu}

\subsection{Funkcionál a formulace úloh variačního počtu}
\subsection{Gteauxova (směrová) derivace, extremály a kritické body}

\section{Posloupnosti a řady funkcí}

\section{Míra a měřitelné množiny}

Systém lebeguovsky a borelovsky měřitelných množin. Míra a úplná míra.

\subsection{$L_p$ prostor}
Definice $L_p$ prostoru, norma, třída ekvivalence, Holderova a Minkovského nerovnost, úplonost $L_p$ prostoru a Rieszova věta, separabilita, konvergence posloupnosti funkcí, Jegorovova věta.

\section{Lebesguevův integrál}

\section{Křivkový a plošný integrál}
\subsection{Křivkový integrál prvního druhu}
\subsection{Křivkový integrál druhého druhu}

\section{Fourierovy řady}
\subsection{}
