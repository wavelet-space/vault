\chapter{Kvantová mechanika (nerelativistiská)}

Pravděposobnostní popis, komutátor, operátor, operátor pozice, operátor hybnosti,  relace neurčitosti, Schrodingerovat rovnice bezčasová 

\section{Příklady k přednáškám }

Poruchový počet


\section{Vodíku podobný atom}
\label{QM.30.01.2021}

Uvažujeme jeden elektron a jádro sestávající z 1 až Z protonů.
Podle klasické fyziky by atom vodíku neměl být vůbec stabilní

Hamiltonián

$$
    H = -\frac{h^2}{2} ...
$$

Jádro berem jako nekonečně těžké


pro konečné hmotnosti jádra


pro limitu $r \to \infty$ je Coulombův potenciál do nuly


Vlnová funkce bude záviset na kvantových číslech z momentu hybnosti
a dodatečné hlavní kvantové číslo $n$


$$
    \psi = \psi_{nlm}(r, \Phi, \phi)
$$


integrály pohybu znamená že se zachovává energi


Řešení Shcrodingerovy rovnice pro atom vodíku

$$

    \hat{H} = \frac{h}{2m} ...
$$


$$
    \hat{L}^2 = -h^2\Delta_{\Phi, \phi}, \quad \hat{L}_z = -ih prt[Phi]
$$


Vlnová funkce pak bude mít tvar


Využijeme substituci

$$
	R(r) = \frac{u(r)}{r} \therefore \ldots
$$

Pro jednoznačnost přejdeme k bezrozměrných jednotek

Zde $a$ je bohrův poloměr

$$
    a = 4\pi\epsilon_0 \frac{h}{me^2}
$$

Také zavedeme berzoměrnou energii


Požadujeme aby vlnová funkce byla normalizovaná
$$
	\int_{0}^{\infty} {|R(r)|}^2 r^2 dr = 1
$$$$
    \eta = \frac{}
$$
Vnejniřším řádu to vede na podmínku $\gamma (\gamma - 1) =  l(l + 1)$

Tím pádem máme dvě možné hodnoty: $\gamma = l + 1$ a \gamma = -l$


Pro tu druhou ale $\rho \to  \infty$ diverguje.

Dosadíme do $f(\rho)$  do rovnice (řady) $\frac{d^2}{d\rho^2} - 2\alpha \ldots$

Má platit pro libovolné $\rho$

pro koeficienty tak musí platit rekurentní vzorec
$$
	a_{\nu + 1} = \frac{2\alpha (\nu + l + 1) - 2Z}{(\nu + l + 2)(\nu + l + 1)-l(l + 1)} a_{\nu}, \quad \nu = 0, 1, \ldots
$$Jak se řada chová pro velmi vysoké indexy?
Pokdu $\nu$ bude dost velké pak bude dominovat a vede to na rozvoj exponenciely.
taková nekonečná řada se dá sečíts a vyjde $f = \rho^{l + 1} e^{2\alpha\rho}$, potom by ale vlnová funkce v limitě $\rho \to \infty$ divergovala a  tedy místo nekonečné řady musíme mít polynom konečného řádu a dostaneme konečnou řadu. Musíme tedy řadu zastavit na konečném indexu $\nu$. Pro jsitou hodnotu $\nu = n_r$ musí být nulový čitatel.
Z toho plyne $2\alpha(n_r + l + 1) = 2Z, \quad n_r = 0, 1, \ldots$

Toto je kvantovací podmínka pro $\alpha = \sqrt{-\epsilon}$

pro bezrozměrnou energii pak dostaneme

$$	
	\epsilon = -\alpha^2 = - \frac{Z^2}{n^2}, \quad n = 1, 2, \ldots
$$Energie vodíku pak vyjde v základním stavu

$$
	E_n = - \frac{1}{4\pi\epsilon_0}  \frac{Z^2e^2}{}
$$Energie $E_n$ v Couloumbickém potenciálu jsou

[obr]

Energie základního stavu není degenerovaná avšak energie všech excitovaných stavů jsou degenerované a ta je rovna $\sum_{l=0}^{} (2l + 1) = n^2$

Celkovou vlnovou funkci poté dostaneme dosazením ... do $f(\rho)$


\subsubsection{Laguerrovy polynom}

Vlnové funkce několika nějnižších stavů jsou
$$
R_{n=1,l=0}(r) = \frac{Z}{}  
$$Pravděpodobnost nalezení elektrony v daném objemovém elementu zadaném $(r, r + dr)$, $\theta, \theta + d\theta)$, $(\varphi, \varphi + d\varphi)$.

Integrací přes prostorový úhel dostaneme pak...

radiální hustoty pravděpodobností $P_{nl}(r) = |R_{nl}|^2r^2$.

S rostoucí energií se pravděpodobnost nalezení elektronu vzdaluje od jádra.

s, p, d stavy

Polární diagram ukazují závislost 


Při přechodu z jednoho stavu do druhého se musí vyzářit jeden foton. Můžeme přecházet diskrítně mezi hladinami a pro každý přechod se přísluší jeden foton.

fe+25, kde se vyskytuje
přechod od rel k nerel?
kvantová elektrodynamika

\section{Spojité spektrum}

\chapter{Metoda Free Electron Molecular Orbita (FEMO)}
\label{QM.07.05.2021}

Ve složitějších látkách neumíme přesně řešit (určit) Schrödingerovu rovnici 
Při předpokladu ,že můžeme zanedbat vliv interakce elektronů, pak můžeme použít metodu FEMO.

Aplikace na $\pi$ elektrony v $1,3$ butadienu.

Máme čtyři elektrony (fermiony), každý y nich může mít spin $\pm \frac{1}{2}$.

Slaterův determinant umožňuje generovat bázové funkce.

Funkce jsou ortonormální.

Vlnové funkce budou kombinacemi funkcí $\psi_1$ a $\psi_2$

Vlnová funkce musáí být anti

Pro efektivní šířku potenciálové jámy použijeme hodnotu $L = 4 \times 1.4 \time 10^{-10} \SI{m}$.

Pro vlnovou délku fotonu 


Pro složitější molekulu např. $1, 3, 5$ \ldots


\subsection{Molekula benzenu}
Metodu FEMO můžeme použít i pro uzavřenou molekulu jako je cyklická, kruhová molekula benzenu.
Benzen má $3$ dvojné vazby a $6-\pi$elektronů

Máme dvě rovnice $ A ¤\sin 0 B \cos 0 = A \sin kL +  \cos KL $ a další rovnice plyne z podmínky spojitosti derivace $ D_t \psi(0) = D_t \psi(L)$

$$
A k  \cos(0) - Bk\sin0 \ldots
$$

a dostaneme maticovou soustavu, pro kterou determinant musí být nula, abychom měli netriviální řešení

To vede na rovnici $\cos kL = 1$ a získáme podmínky 

$$
kL = 2n\pi\ quad n = 0, \pm 1, \pm 2, \ldots
$$

Energie bude 

$$
E_n = {(2n)^2 h^2}{8mL^2}
$$

Jaké jsou vlnové funkce?

$$
\begin{gather}
    \psi_0 = \sqrt{\frac{1}{L}}    
\end{gather}
$$
