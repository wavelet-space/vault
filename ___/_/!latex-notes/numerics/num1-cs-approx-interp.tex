\section{Úvod}
Základním principem numerických metod je diskretizace proměnných a to buď prostorových (\textit{spatial}) anebo časových (\textit{temporal}) proměnných.

\begin{itemize}
    \item Approximation
        \begin{itemize}
            \item Aproximace na okolí bodu
            \item $L_2$ approximace
            \item Interpolation $\R \mapsto \R$
            \begin{itemize}
                \item Použijeme pokud chceme danými body proložit polynom tzn, že aproxumace přesně prochází body.
                
                \item Lagrange Interpolation Method
                    Lagrange Interpolating Polynomial
                \item Newton Interpolation Method
                    Newton Interpolating Polynomial
                \item 
            \end{itemize}
        \end{itemize}
    \item Iterativní metoda
    
    \item Přímá metoda

    \item Numerická derivace
    \item Numerická integrace
        \begin{itemize}
            \item   - Newtonovy-Cotesovy vzorce a složené Newtonovy-Cotesovy vzorce,
  - Rombergova kvadratura
  - Gaussova kvadratura.
        \end{itemize}
\end{itemize}

\chapter{Approximation and Interpolation}

Mějme funkci $f: \R \mapsto \R$ definovanou v $n + 1$ bodech $x_i$.

\subsection{Aproximace na okolí bodu}

Aproximace funkce $f$ na okolí bodu $x_0$ pomocí Taylorova polynomu $T$. Předpokldáme, že funkce $f$ má v daném bodě $x_0$ alespoň derivaci $n$-tého řádu (včetně) jejichž hodnoty známe. Hledáme funkci $\varphi$, která má v daých bodech stejnou derivaci a tedy je dosti podobná dané funkci $f$.

\begin{equation}
    \varphi^{(k)}(x_0) = f^{(k)},  \quad k = 0, 1,\ldots n,
\end{equation}

Toto chování je zajištíme pomocí Taylorova polynomu. 


\begin{equation}\label{eq:Taylor}
T_n^{f,a}(x) = \sum_{k=0}^n  \frac{f^{(k)}(a)}{k!} (x - a)^k
\end{equation}


rozepsáno 

\begin{equation}
T_n^{f,a} = \frac{f^{(0)}(a) (x - a)^0}{0!} + \frac{f^{(1)}(a) (x - a)^1}{1!} + \ldots + \frac{f^{(n) (x - a)^n}}{n!}
\end{equation}

Chyba aproximace $error(x) = f(x) - T_n(x) = f^{n+1}(\xi) \frac{(x - x_0)}{(n + 1)!}, \xi \in U(x_0) $

\begin{example}
Aproximujte funkci $f := \sin(x)$ (se středem) v bodě $x_0 = 0$ pomocí Taylorova polynomu čtvrténo stupně $T_4^f(x)$. 

Řešení:  $T_4^f(x) = \sin(0)  + \cos(0)x - \frac{\sin(0) x^2}{2!} - \frac{\cos(0) x^3}{3!} + \frac{\sin(0) x^4}{4!} = x - \frac{x^3}{6}$. Chyba je ve tvaru $error(x) = \frac{\lvert\cos(0) x^5\rvert}{5!} = \frac{\lvert{x^5}\rvert}{120}$
\end{example}

\subsection{
Aproximace v zadaných bodech
%Lagrange Interpolating Polynomial
}

Označím hledaný Lagrangeův polynom stupně $n$ jako $L_n(x)$ a požadujme splnění interpolační podmínky 

\begin{equation}
    L_n(x_i) = f(x_i), \quad i = 0, 1,\ldots, n.
\end{equation}

Podmínka říká, že hledaný polynom $n$-tého stupně $L_n$ prochází přesně body dané fukce $f$ pro dané hodnoty $x_i$. Polynom $L_n$ hledáme ve tvaru 

\begin{equation}
    L_n(x) = \sum_{i=0}^0 f(x) l(x),
\end{equation}


\section{Numerical Analysis}
\begin{itemize}
    \item Rounding Error
\end{itemize}

\textbf{Jednokroková metoda} --

\textbf{Vícekroková metoda} --


\section{Metoda konečných diferencí}

Metoda konečných diferencí je založena na diskretizaci prostoru a času studované domény. Obvykle diskretizujeme prostor ekvidistantně. příklad si ukážeme na rovnici vedení tepla  v jedné prostorové a jedné časové dimenzi. Výslední výpočetní mříž bude tedy dvojrozměrná.