\part{Matematické Modely} 

Na se vyučuje v rámci předmětů:

\begin{enumerate}
    \item Úvod do matematického modelování (NMNM334), Katedra numerické matematiky
    \item 
\end{enumerate}

Dostupná skripta a literatura:

\begin{enumerate}
    \item https://www2.karlin.mff.cuni.cz/~feist/Mod.pdf
\end{enumerate}

Odvození rovnic popisujících proudění:
Základní představy o tekutinách a způsob popisu jejich pohybu, 
* věta o transportu 
* zákony zachování (hmoty, hybnosti a energie) a jejich formulace ve tvaru diferenciálních rovnic, konstitutivní a reologické vztahy, Eulerovy a Navierovy-Stokesovy rovnice, termodynamické zákony.

*Okrajové úlohy teorie pružnosti
  - Tenzor napětí, podmínky rovnováhy, 
  - tenzor konečné deformace, 
  - tenzor malých deformací, 
  - zobecněný Hookův zákon, 
  - Laméovy a Beltramiovy-Michellovy rovnice, 
  - základní okrajové úlohy pružnosti.

* Modelování nevazkého neviřivého proudění
  - nevazké nevířivé proudění popsané pomocí potenciálu rychlosti, 
  - Bernoulliho rovnice, potenciál rychlosti, 
  - úplná potenciální rovnice, její vlastnosti, 
  - okrajové podmínky, formulace úloh pro potenciál rychlosti, 
  - obtékání profilu, síla působící na profil.

* Modelování proudění v porézním prostředí
 - Zákon zachování hmoty v proudění se zdroji
 - Darcyho zákon, formulace úlohy prosakování s nespojitou permeabilitou, 
 - slabá formulace úlohy pro eliptickou rovnici s nespojitými koeficienty.

* Transportní procesy:
 - Rovnice pro šíření koncentrace příměsí v proudící tekutině,
 -  konvektivně difuzní procesy, aplikace v ekologii.