
\section{Atom}

\subsection{Bohrův model atomu}

\subsection{Magnetické vlastnosti}
Zeemanův jev


\section{Molekula}

\subsection{Symetrie molekuly}

\subsection{Pohyb a energie molekuly}

Pro dvouatomovou molekulu můžeme uvažovat vibrační , rotační  a translační pohyb.

Pro vibrační pohyb lze pracovat s Lenard-Jonesovým, Morseovým nebo parabolickým (harmonickým) potenciálem.


Pro vibrační pohyb dvoutaomové molekulu můžeme použít 
Pro rotační pohyb můžeme uvažovat 1) tuhý rotátor tzn. rychlost rotace nemá vliv na vzdálenost mezi atomy.
To platí jen do určité míry, ale pro nízké 

\subsection{Měrné teplo}

Debeyův a Einsteinův model

Pokud roztočím molekulu, můžu pozorovat i vibrace tzn rotačně vibrační stavy. Tyto stavy jsou kvantovány. Pokud na ni posvítím, můžu pozorovat absobční nebo při chlazení emisní rotačně-vibrační spektra.

\subsection{Elektronová struktura molekul, vazby, LCAO}

Co drží molekulu pohromadě? Zřejmě interakce elektronů mezi sebou a jaké vazby mezi sebou vytvářejí.

Vazby známe kovalentní při které dochází ke sdílení elektronů (překryv hustot elektronů), iontová vazba funguje jako předání elektronu mezi atomy, kdy vzniknou opačně nabyté ionty, které se přitahují např. při interakci \ce{Na} a \ce{Cl} $\imply$ \ce{Na+} a \ce{Cl+}
S kovalentní vazbou je spojena elektronegativita, s iontovou vazbou zase afinita a excitace. 

Pokus o vysvětlění vodíkové vazby Heitler a London (1927).
Magnetické vlastnosti \ce{N2} a \ce{O2} se liší! nenulová magnetická složka spinu indukuje magnetismus viz kyslík.

Pauliho vylučovací princip  Hundova [:Handova:] pravidla.

Linera Combination of Atomic Orbitals (LCAO) aproximace. Ve výchozím stavu máme dva vodíkové atomy daleko od sebe a poté je přibližujeme.
Hledáme elektonovou strukturu molekuly $H2+$.


Hybridizace a překryv orbitalů. Uhlík a hybridizace a tvorba čtyřvazných molekul

Bor...