\chapter{Měrné teplo ideálního krystalu}

Tepelní kapacita za konstantního tlaku (izochorická) pro makroskopickou soustavu.

\begin{equation}
    C_V = \pdv{U}{T}, \quad [V]
\end{equation}

Pro měrné teplo ideálního krystalu existuje Einsteinovo (1906) a Debeyovo (1912) teoretické odvození. Oba modely vycházejí z Planckova principu kvantování energie.  

\section{Empirické vlastnosti tepelné kapacity}

\begin{enumerate}
    \item Při nulové teplotě je $\lim_{T \to 0} C_V = 0$.
    \item Při nízkých teplotách do 15K je $C_V = konst. T^3$.
    \item Při vysokých teplotách v blízkosti bodu tání je $ \lim_{T \to \infty} C_V = 3R$ Dulong-Petitův zákon (klasický model).
\end{enumerate}

\section{Einsteinův kvantový teoretický model}

Předpoklady:
\begin{itemize}
    \item \textbf{Nezávislost} vibrace jedné částice je nezávislá na vibracích okolních částic.
    \item \textbf{Izotropie} vibrace, tj. žádný směr vibrace částice není preferován, všechny mají stejnou váhu -- jsou stejně pravděpodobné.
    \item \textbf{Harmonicita} vibrací tj. vibrace částice se dá popsat harmonickou funkcí a všechny částice kmitají na stejné frekvenci $\nu$.
    \item Počet oscilátorů je $3N_A$ molů, jde $N_A$ je Avogardova konstanta.
    \item  Spektrum energií není spojité, ale kvantované v jednotkách $h\nu$ 
\end{itemize}

Z předpokladu o nezávislosti odvodíme, že partiční funkce vibrací $Q_{vib}$ je rovna součinu partičních funkcí jednotlivých částic $q$ \cite{Malijevsky2009}. Pro krystal o $N$ částicích je tedy
$$
    Q_{vib} = q^N.
$$

Z předpokladu o izotropnosti plyne, že partiční funkce jedné částice 
$$
    q = q_x . q_y . q_z = q_x^3.
$$.

Z předpokladu o harmonicitě plyne, že pro vibraci částice (molekyly) v kvantovém stavu (s kvantovým číslem) $n$ je energie $ \epsilon_n = h \nu (n + \frac{1}{2}) = \hbar \omega (n  + \frac{1}{2})$ .

$$
q_x = \sum_{n = 0}^{\infty} e^{ \frac{-\epsilon_{\n}}{k_B T}}.
$$

Sečteme geometrickou řadu pro $q_x$ a další odvození celkové partiční funkce viz \cite{Malijevsky2009}.

V \cite{Danis2019} je odvození s pomocí pravděpodobností, že částice je ve stavu $n$ máme $p_n \sim e^{-\frac{-\epsilon_n}{k_B T}}$.

% % todo Boltzman, Raoult, Avogadro, Planck Macro

Střední energii získáme derivováním partiční funkce podle $\beta = \frac{1}{k_B T}$. 
$$
\bar E = \pdv{Q}{\beta}
$$

Výraz pro izochorickou měrnou tepelnou kapacitu získáme derivováním střední energie podle teploty.

$$
C_V = \pdv{\bar E}{T} = \ldots
$$

Nakonec dostaneme Einsteinův výraz pro tepelnou kapacitu ideálního krystalu

\begin{equation}
    C_V_{Einstein}  = 3 N_A k_B\ (\beta / T)^2 \frac{e^{-\beta/T}}{ (1-e^{-\beta/T})^2 }
\end{equation}

, kde $\beta = h\nu/k_B = \hbar \omega/k_B, \quad [K]$ je Einsteinova charakteristická teplota.

\subsection*{Limitní případ pro $T \to 0$}

Pro nízké teploty bude výraz $e^{-\beta/T}$ velmi malý a dostaneme

$$
    C_V = 3N_A k_B \Big(\frac{\beta}{T}\Big)^2 e^{-\beta/T}
$$

\subsection*{Limitní případ pro $T \to \infty$}
 
 Pro vysoké teploty bude $\frac{1}{k_B T} \to 0$ a tedy $\beta/T$ je malé; exponencielu nahradíme Taylorm $e^x = 1 + x$
 
 $$
    C_V = 3N_A k_B = 3R
$$
, kde $R$ je univerzální plynová konstanta.  

\subsection*{Závěr} 

Model dobře popisuje model při vyšších teplotách a nulové teplotě, avšak pro teploty blízké absolutní nule neodpovídá pozorované závislosti $C_V = \alpha T^3$ tj. nesplňuje asymptotický vztah.

\section{Debeyův kvantový teoretický model}

Debeyův model již nepředpokládá, že částice vibrují nezávisle a na stejné frekvenci tzn. že počítají s elastickými vazbami. Odvození vzrorce viz \cite{Danis2019}.

\begin{equation}
    C_V_{Debeye} = 9N_Ak_B \Big(\frac{T}{\theta_D}\Big)^3 \int_{0}^{\frac{\theta_D}{T}}  \frac{x^4 e^4}{(e^x - 1)^2} dx.
\end{equation}

\subsection*{Limitní případ pro $T \to \infty$}

Horní integrandu mez je malá a tak nahradíme exponenciálu v integrandu Taylorovým polynomem.

$$
C_V \approx 3N_A k_B
$$
, což odpovídá Dulong-Petitovu zákonu.

\subsection*{Limitní případ pro $T \to 0$}

Horní mez integrálu bude nekonečno a vzorec přejde na
$$
C_V \approx T^3
$$

\subsection*{Závěr}

Debeyeův model splňuje asymptotické chování, platí pro kove i nekovy, ovšem za vysokých teplot již se projevuje vliv volných elektronů, které musíme opět započítat. Debeyův model je tak lepším přiblížením než Einsteinův.

\begin{thebibliography}{}

\bibitem{Danis2019} S. Daniš: Atomová fyzika a elektronová struktura látek. MatfyzPress Praha, 2019.
\bibitem{Malijevsky2009} A. Malijevský: Lekce ze statistické termodynamiky, VŠCHT Praha, 2009
\bibitem{Kraus2017} I. Kraus., J. Fiala : Elementární fyzika pevných látek, ČVUT Praha, 2017
\end{thebibliography}