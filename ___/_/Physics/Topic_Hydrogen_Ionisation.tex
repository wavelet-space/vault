
\subsection*{Ionizační energie elektronu vodíkového atomu}

\textbf{Určete ionizační energii elektronu pro základní stav vodíkového atomu a vyjádřete tuto energii v eV.}

\vspace{0.5cm}

Vyjdeme z Bohrova modelu atomu, který je postaven na postulátech:
\begin{enumerate}
    \item Elektron se může pohybovat po kruhových drahách (orbitech), takových, že 
    velikost momentu hybnosti elektronu $L = r p = r m_e v$ je celistvým násobkem $\bar h$ tj. $L_n = n \bar h = n\frac{h}{2\pi}$, kde $n = 1, 2, 3, \ldots$ je hlavní kvantové číslo.
    
    \small{Pro vlnovou délku platí de Broglieho vztah $\lambda = \frac{h}{p}$ a pro délku orbity $2\pi r = n \lambda = \frac{nh}{p} \therefore pr = n\hbar = L$}
    
    \item Elektron pohybující se po dovolených drahách nevyzařují energii.
    \item Elektron může přecházet pouze z jedno povolené orbity na jinou a při tomto přechodu vyzáří nebo pohltí příslušné množství energie o frekvenci $\nu  = \frac{E_f - E_i}{n}$. 
\end{enumerate}


Pro celkovou energii elektronu platí

\begin{equation}
    E = E_k + Ep = \frac{1}{2} m_e v^2 - \frac{e^2}{4\pi\epsilon_0 r^2}
\end{equation}

Kde druhý člen reprezentovaný Coulombovskou silou, představuje dostředivou sílu a tedy.

\begin{equation}\label{eq.2}
    \frac{e^2}{4\pi\epsilon_0 r^2} = m_e \frac{v^2}{r} \therefore m_e v^2 = \frac{e^2}{4\pi\epsilon_0 r}
\end{equation}

Dosadíme li do vzorce pro energii dostaneme:

\begin{equation}
    E = -\frac{e^2}{8\pi \epsilon_0 r}  =  -\frac{1}{2} m_e v^2
\end{equation}

Podle prvního postulátu o kvantování momentu hybnosti je $v = \frac{n\bar h}{m_e r}$ a tedy dosazením za rychlost a úpravou \ref{eq.2} získáme poloměr 

\begin{equation}
r_n = \frac{4\pi \hbar^2 \epsilon_0}{e^2 m_e} n^2
\end{equation} 

$r_1 \equiv a_0 \approx 0.53 A$ je tzv. Bohrův poloměr. Kvantování momentu hybnosti tedy způsobuje i kvantování poloměru orbitu. Tento poloměr opět můžeme dosadit do vzorce o energii a dostaneme

$$
    E_n = -\frac{e^4 m_e}{32\pi^2 \hbar^2 \epsilon_0^2} \frac{1}{n^2}
$$
a pro $n = 1$ je $r1$ poloměr základního stavu elektronu vodíkového atomu, kdy je elektron nejtěsněji vázán k jádru.

\begin{equation}
    E_1 = -\frac{e^4 m_e}{32\pi^2 \hbar^2 e_0^2} = 2.17 . 10^{-18} J = 13.6 eV.
\end{equation}

kde $1J = 6.241509⋅1018 eV$ a Rydbergova konstanta $Ry = \frac{e^2 m_e}{2\hbar^2}$.

\vspace{0.5cm}

Ionizační energie vodíkového atomu je energie potřebná k přemístění elektronu ze základního stavu do vzdálenosti $r \to \infty$ a je tedy rovna energii základního stavu $13.6 eV$.




