\part{Termodynamika a statistická fyzika}

Zde vnzikající text  slouží jako poznámky k předmětu Teorie tekutin a směsí a pokrývají základy termodynamiky a statistické fyziky, nutné pro pochopení látky, ale značně ji přesahují. Proto snad budou použitelné obecně, jako základní přehled tohoto tématu. Následuje seznam otázek, termínů (hesel), která je třeba definovat, vysvětlit a zasadit do kontextu.

\begin{enumerate}
    \item termodynamický děj
    \item termodynamický proces
\end{enumerate}

\begin{thebibliography}{}
\bibitem{Callen1985} Callen, 1985
\end{thebibliography}

\chapter{Rovnovážná termodynamika}

\section{Makroskopická vs mikroskopické pozorování}

Makroskopický popis určuje vlastnosti systému z makroskopicky měřitelných veličin.

Statisticý popis..

Přenost tepla, probíhá okamžitě, např. při zahřátí tělesa, tj. dodání tepla, se toto projeví v celém objemu tělesa najednou. V klasické rovnovážné termodynamice zanendbáváme čas a tudíž změna působí na systém okamžitě.

Rovnovážná termodynamika \textit{equlibrium thermodynamics} se zabývá vratnými procesy, proto také někdy mluvíme o vratných termodynamických procesech a vratně termodynamice (\textit{reversible}).

V této kapitole zavedeme klasicky (fenomenologicky) základní principy, postuláty a zákony na kterých stojí rovnovážná termodynamika. Způsob, jakým byly tyto zákonitosti objevovány a jak jsou popsány v učebnicích je velmi různorodý. Myslím, že pro důkladné pochopení se vyplatí podívat se do vícero zdrojů, které na problém hledí z různých hledisek. Pro začátek doporučím knihy \cite{Callen1985}.

\section{Systém, hranice a okolí}

Jako \term{systém} $\mathfrac{S}$ označíme oblast $\Omega$ prostoru ($E^3$), oddělený od \term{okolí} skutečnou nebo myšlenou hranicí $\partial \Omega$, jehož vlastnosti chceme popisovat. Podle vlastností hranice rozlišujeme následující druhy systému.

\begin{enumerate}
    \item \textbf{Otevřený} systém může vyměňovat s okolím energii a hmotu.
    \item \textbf{Uzavřený} systém  může vyměňovat s okolím pouze energii.
    \item \textbf{Izolovaný} systém nemůže vyměňuje s okolím hmotu ani energii.
\end{enumerate}

Uvedená hranice systému se také označuje jako stěna (\textit{wall}). Jkao skutečnou hranici si můžeme představit nádobu. Jako myšlenou hranici např. povrch krystalu v tavenině,

Stěna může být 
\begin{enumerate}
    \item \textbf{Adiabatická} stěna
    \item \textbf{Diatermická} stěna
\end{enumerate}


\section{Stav termodynamické rovnováhy systému} 

Rovnovážná termodynamika popisuje stavy systému, ve kterých se makroskopicky pozorovatelné veličiny v čase (zdánlivě) nemění a jsou tak rovny časově středním hodnotám. Makroskopický systém je ve stavu termodynamické rovnováhy, pokud se žádné makroskopicky pozorovatelné (pozorované) veličiny v čase nemění. Mluvíme o kvazistatických procesech, kdy systém přechází z jednoho rovnovážného stavu do dalšího  a tím se mže dostat daleko od původního stavu skrze sekvenci rovnovážných stavů jen malou změnou podmínek.

\textbf{Postulujeme}, že každý izolovaný systém dospěje při $t \to \infty$ do stavu rovnováhy, tj. stav, ve kterém se již jeho makroskopické vlastnosti v čase nemění. tento stav je určen pouze vnitřními faktory a nikoliv např. dříve působícími vnějšími vlivy (silami). Systém ve stavu termodynamické rovnováhy není závislý na předchozích stavech tj. na historii. Např. pokud vhodíme kostku ledu do sklenice a necháme ji v místnosti dostatečně dlouho, tak led roztaje a systém se dostane do rovnováhy s okolím. V tomto stavu nejsme schopni rozpoznat z jakého stavu do rovnováhy systém došel. Bylo to tání ledové kostky nebo prosté přilití vody do sklenice? Obecně mají fyzikální systémy paměť, kdy jejich vývoj je odezvou na předchozí stav systému a interakci s okolím. $\imply$ vývoj systému je nevratný na rozdíl od systému bez paměti.  Systémy jsou symetrické vzhledem k otočení šipky času.

\subsection{Popis rovnovážného stavu pomocí makroskopicky měřitelných veličin}

závislé a nezávislé stavové veličiny, stavová rovnice

\section{Teplota, energie a entropie}

\section{Postuláty klasické termodynamiky}

Jako jednoduchý systém označíme takový, který je popsatelný několik málo makroskopicky měřitelnými parametry jako je tlak, objem, teplota.  

\begin{theorem}
Pro jednoduchý systém existuje stav nazývaný rovnovážný stav takový, že ho lze, z makroskopického pohledu, popsat vnitřní energií $U$, objemem $V$ a molárním množstvím $N = \sum N_i$ jeho chemických komponent.  
\end{theorem}

Dále uvedeme postuláty o maximu entropie.

\begin{theorem}
Pro každý rovnovážný stav existuje funkce extenzivních parametrů nazývaná entropie $S$ taková, že \ldots 
\end{theorem}

Pokud známe entropii systému jako funkci extenzivních parametrů, pak všechny jeho termodynamické vlastnosti z ní můžeme odvodit. Proto tomuto vztahu říkáme \textit{fundamentální vztah}.

\textbf{Entropie je zavedena pouze pro rovnovážný stav!}

Tato formulace postulátů je převzata z \cite{Callen1985}.

\section{Zákony klasické termodynamiky}

Klasická termodynamika zavádí (postuluje) tzv. termodynamické zákony (též postuláty) které považujeme za platné a souhlasící s experimentem. Někdy jsou též nazývané \textit{větami}, což by ale naznačovalo, že je lze nějak dokázat, proto je budeme nazývat zákony nebo též postuláty. Známe tzv. tři termodynamické zákony, ale předchází dodatečně formulovaný nultý zákon, o kterém je také dobré něco vědět. 

\subsection{0. termodynamický zákon}

(\textit{zero law of thermodynamics})
    
\subsection{1. termodynamický zákon}

(\textit{first law of thermodynamics})

\subsection{2. termodynamický zákon}

(\textit{second law of thermodynamics})

\subsection{3. termodynamický zákon}

(\textit{third law of thermodynamics})

0. termodynamický zákon: existence empirické teploty jako intenzivní veličiny

1. termodynamický zákon: existence vnitřní energie 

$$
    U = Q + W 
    dU = \delta Q + \delta W
$$

Pod symbolem práce se skrývají různé formy např. mechanická, chemická, elektromagnetická.

2. termodynamický zákon: existence entropie $S$
3. termodynamický zákon: $\lim_{T \to 0} S = 0$

\section{Termodynamický proces a rovnováha}

\section{Fyzický model systému v termální rovnováze}

Poznámky k [Málek a Souček, 2020]

Teorie interakcí v kontinuu se dá rozdělit na dva přístupy

1. všechny komponenty směsi spolu koexistují [Truesdell, Toupin] viz \textit{continuum mixture theory}
2. [Drew Passman] \textit{multi phase theory}
    Jak to souvisí s phase-field metodou (otherwise called diffuse interface)

\begin{itemize}

\item \textbf{Princip minimalizace energie $E$} -- rovnovážná hodnota nijak neomezeného vnitřního parametru je taková, že \textit{minimalizuje vnitřní energii} pro danou celkovou entropii.

\item \textbf{Princip maximalizace entropie $S$} -- rovnovážná hodnota nijak neomezeného vnitřního parametru je taková, že \textit{maximalizuje entropii} při dané celkové vnitřní energii.

\end{itemize}

\subsection{Konkavita entropie $S$}

\subsection{Konvexita energie $E$}

\subsection{Entropie při složení dvou systémů}

$$
S^C = S^{A + B} = S^A + S^B 
$$

\section{Jednosložková jednoduchá tekutina}
jednosložková neboli jednokomponentní

NVE

Jak se zvýší entropii jednosložkového systému?

\section{Vícesložková jednoduchá tekutina}
vícesložková neboli vícekomponentní uniformní složená z komponent

E

\section{Extenzivní parametr}

\section{Intenzivní parametr}


\textbf{Entropie a Energie nejsou přímo měřitelné, zatímco T, P, V ano.}


% \subsection{van der Waals Equation}

% Calculate the constants (parametres) $a$ and $b$ for $\ce{H2O}$ in the van der Waals and in the Redlich-Kwong equations. 

% Vezmeme kritické hodnoty z tabulek a pozor na jednotky R.

% \begin{equation}
% \left( P + \frac{n^2a}{V^2} \right) \left( V - nb \right) = nRT
% \end{equation}

% For 1 mole the equation simplifies to 

% \begin{equation}
% \left( P + \frac{a}{V_m^2} \right) \left( V_m - b \right) = RT
% \end{equation}

% % http://www2.ucdsb.on.ca/tiss/stretton/database/van_der_waals_constants.html
% % The constants a and b are called  van der Waals constants. They have positive values and are characteristic of the individual gas. If a gas behaves ideally, both a and b are zero, and van der Waals equations approaches the ideal gas law PV=nRT. 
% % a = 5.537 -- provides a correction for the intermolecular forces.
% % b = 0.03049 --adjusts for the volume occupied by the gas particles.

% \subsection{Redlich-Kwong Equation}

% \section{Two Point Eutectic System}
    
% \subsection{Congruent Melting and Eutectic Point}

% \chapter{Příklady z klasické chemické termodynamiky}

% Uvažujem stále množství ideálního plynu při výchozí teplotě \SI{20}{\celsius}


% % Let $S$ be an isolated thermodynamical system with constatnt energy E (U?), consisting of $N$ particles, where N is a sufficiently large number, so that it is not possible to know the energy of a individual particle. 

% % We can only know the  
% % \subsection{Helmholz Free Energy}
% % \subsection{Gibbs Free Energy}
% % \begin{equation}
% %     G = H -TS    
% % \end{equation}

% % \begin{equation}
% %     dG = dH - sdT - TdS
% % \end{equation}

% % Gibbs free energy is a thermodynamic potential that can be used to calculate the maximum reversible work that may be performed by a thermodynamic system at a constant temperature and pressure.

\section{Maxwellovy rovnice}

\section{Gibbsovy-Helmholtzovy rovnice}

\chapter{Statistická termodynamika}

% \subsection{Boltzmann Distribution}

\chapter{Kinetická teorie plynu}

\section{Model ideální plynu}
\subsection{}

Mějme plyn skládající se z jednoho druhu hmotných částic o stejné hmotnosti $m$, které spolu neinteragují jinak než pružnou srážkou tzn. že při srážce se jen změní směr rychlosti částice. Částice mají stejnou rychlost.
Entropie izotermální expanze ideálního plynu.
Entropie izotermální komprese ideálního plynu.


Otázky pro B. Gaše: Dobrý den, 

První termodynamický zákon se uvádí v podobě $ U = Q + W $, nebo diferenciálně $ dU = dQ + dW $, kde $Q$ značí teplo a $W$ práci. V různých učebnicích ale různě uvádí,  co vše se zahrnuje pod $W$, např., 
mechaniská, elektrická, magnetická, chemická. Jaké všechny členy tedy pod $W$ zahrnout, pokud by měl výčet být úplný?

Na ideální plyn klademe tyto předpoklady:

\begin{itemize}
    \item 
\end{itemize}

\chapter{Nerovnovážná termodynamika}

