
\documentclass[a4]{article}

\usepackage[utf8]{inputenc}
\usepackage[utf8]{inputenc}
\usepackage[czech]{babel}
\usepackage{amsmath} 
\usepackage{amsthm}
\usepackage{amssymb}
\usepackage[acronym]{glossaries}
\usepackage{hyperref}
\usepackage{siunitx}
\usepackage{physics} % \vb bold vector http://mirrors.ibiblio.org/CTAN/macros/latex/contrib/physics/physics.pdf
\usepackage{braket}
% $\braket{\varphi|\psi}$
% $\bra{\varphi}$
% $\ket{\psi}$
% $\ket{\varphi}\bra{\psi}$
% $\ketbra{\varphi}{\psi}$
\usepackage[hscale=0.7,vscale=0.8]{geometry}
\urlstyle{same}
\usepackage{mhchem}
\def\R{\mathbb{R}}
\def\N{\mathbb{R}}


\newtheorem{theorem}{Theorem}


\newcommand{\cvec}[1]{\ensuremath{\begin{pmatrix}#1\end{pmatrix}}}

% ====================================================================
% Mathematics
% ====================================================================
\newcommand\N{\ensuremath{\mathbb{N}}}  % Set of natural numbers
\newcommand\R{\ensuremath{\mathbb{R}}}  % Set of real numbers
\newcommand\Z{\ensuremath{\mathbb{Z}}}  % Set of integer numbers
\renewcommand\O{\ensuremath{\emptyset}} % Empty set
\newcommand\Q{\ensuremath{\mathbb{Q}}}  % Set of rational numbers
\newcommand\C{\ensuremath{\mathbb{C}}}  % Set of Complex numbers
% Exponenciel function
\newcommand{\e}[1]{\ensuremath{ {\rm e}^{#1} }}

\theoremstyle{definition}
\newtheorem{definition}{Definition}[section]

% ====================================================================
% Physics
% ====================================================================

% Continuum Mechanics
\newcommand\Stress{\ensuremath{\mathbf{\sigma}}} % stress
\newcommand\Strain{\ensuremath{\mathbf{\epsilon}}} % strain

%Numbered environment
\newcounter{example}[section]
\newenvironment{example}[1][]{\refstepcounter{example}\par\medskip
   \noindent \textbf{Example~\theexample. #1} \rmfamily}{\medskip}



\DeclareMathOperator{\Res}{Res}

\title{Thermodynamics, Statistical and Quantum Mechanics}

\begin{document}

\section{Cahn-Hilliardova rovnice a její odvození}

Rozbor článku: Physical, mathematical, and numerical derivations of the Cahn–Hilliard
equation.

V článku se odvodí fyzikálně, matematicky a numericky Cahn-Hilliardova rovnice pro binární směs.
Touto rovnicí lze popsat odmíšení fází (exsoluci) v binární směsi, kdy se původně homogenní směs
tvořená jednou fází rozdělí dvě fáze v podobě domén obahující jednu komponentu.

CH popisuje časový vývoj konzervativního pole, které je popsáno spojitou a dostatečně diferencovatelnou funkcí polohy (a času).
Oddělení fází je způsobeno vlivem ne Fickovské (\textit{uphill}) difuze, která je řízena nikoliv gradientem koncentrace, ale  gradientem chemického potenciálu.

CH byla původně navržena jako model pro jev nazývaný spinodální dekompozice, pozorovaná v binárním A-B systému (slitině) při konstantní teplotě. 

Počáteční stav systému je, že se skládá z určitého molárního množství fáze B a spontánně se rozdělí na 
dvě fáze A a B, které mají stejnou krystalovou orientací (fluida?) ale různé složení popsané 
kompozičním (koncentračním?) polem $c(\vb, t)$.

Časový vývoj $c$ je popsán řešením CH.

$$
\partial_t c = \div(M \grad{\mu}) = M \laplacian{\mu}
$$

, kde $M$ je konstantní mobilita (difuzivita), která obecně je tenzorem (2. řádu) a 
$\mu$ je lokální chemický potenciál definovaný jako funkce

$$
    \mu = F(c) - \kappa \laplacian{c}.
$$

$F(c)$ je hustota Helmholtzovy volné energie pro jednu molekulu (částici) homogenního systému
o složení $c$ a $\kappa > 0$ je konstanta nazývaná také jako koeficient gradientu energie $\kappa = \epsilon^2$, který úzce souvisí s mezifázovým rozhraním.

Základním konceptem (vlastností) CH je, že rozhraní mezi fázemi $\alpha$ a $\beta$ není ostré (nulové tloušťky), ale že má konečnou nenulovou tloušťku, kde $c$ se mění postupně.

Z obrázku [1] můžeme vidět, že složení $c$ a v termodynamicky rovnovážném stavu je uvedeno pro každou doménu jako $c^{eq}_{\alpha}$ respektive $c^{eq}_{\beta}$.

Na rozhraní je potom $c_{\alpha} c < c_{\beta}$.

Výhodou použití CH oproti jiným metodám je v tom, že nemusím explicitně sledovat rozhraní mezi fázemi, protože to 
je zahrnuto jako vlastnost (hodnota) pole ($c(\vb, t)$). Tomuto konceptu se také říká difuzní rozhraní.

Proto můžeme CH výhodně použít např. pro vývoj mikrostruktur obecně, kde se předtím pracně aplikovaly metody ostrého rozhraní označované jako 
Sefanův problém, kdy se rozhraní mezi fázemi časově proměnné 

%(https://en.wikipedia.org/wiki/Stefan_problem).
% (https://www.researchgate.net/figure/Schematic-of-1D-Stefan-problem_fig1_315382305)

V modelech fázového pole poté zavadíme parametr uspořádání (order parameter) či fázoové pole $\phi(\vb, t)$ namísto 
kompozičního (koncentračního) pole $c(\vb x, t)$, 

\section{Fyzikální odvození z termodynamických principů}

Fyzické odvození je provedeno z termodynamických principů, kdy volná energie izotropního systému oblasti $\Omega$  

\begin{enumerate}

\item Odvodíme hustotu Helmhotzovy volné energie $F(c)$ pro homogenní systém.

\item Odvodíme lokální hustotu Helmholtzovy energie pro nehomogenní systém v podobě funkcionálu $c(\vb x, t)$

\item Lokální chemický poetnciál $\mu$, který musí být potom rovnoměrný v celém systému v rovnováze, je poté definován jako variace (variační) derivace funkcionálu $F$ a gradient 
toku hmoty je $J = - \grad {\mu}$

\item CH nakonec získáme substitucí vztahů mezi $J$ a $\mu$ do rovnice (kontinuity) zachování hmoty.

\ite,m  NVR X½FðcÞ þ 0:5�2jrcj2�dx. 



\end{enumerate}

\subsection{Volná energie homogenního systému}

\subsection{volná energie heterogenního systému}

\section{Matematické odvození CH jako gradientního toku}

[TODO]

\end{document}