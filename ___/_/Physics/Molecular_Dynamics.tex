(Molecullar Dynamics 1. 3.)

\section{Ekvipartiční teorém}

EKvipartiční teorém, pravděpodobnostní rozdělení energie.
Ekvipartice kinetické energie kB * T / 2 per stupěň volnosti
V ekvilibriu budou mít mt různé částice průmerně stejnou moment energie

Ekvipartiční teorém (zákon stejnoměrného rozdělení) z klasické statistické termodynamiky, spojuje teplotu systému a 
pruměrnou energii. 

Střední kinetická energie Ek jedné jednoatomové molekuly bude $N_A$-krát menší než je kinetická energie 
jednoho molu plynu.

\begin{equation}
    E_k = \frac{3}{2} k_B T
\end{equation}
    
V nejjednodušším případě je ekvipartiční teorém vyhádřen pro jednoatomový plyn.
Každá jednotlivá  molekula plynu má tři stupně volnosti a její pohyb je chaotický.
Každému jednotlivému stupni volnosti jednoatomové molekuli příluší střední hodnota kinetické energie o hodnotě  $\frac{1}{2} k_B T$

Pokud spojíme dva  atomy vazbou, uzmem původně 6- stupňům volnosti 1. avšak dvouatomolová molekula má kromě translace nyná také  rotaci, přesně řečené 3 stupně volností pro translaci a 2 stupně 

\begin{equation}
    E_k = E_t + E_r \frac{5}{2} k_B T
\end{equation}

\section{Termostat}

Co je to termostat?
Jaké termostaty máme k dispozici?

\subsection{Nosé Hoover Termostat (SM 4.8)}

Je deterministický a jeden z nejvíce accurate a efficient metoda (algoritmus) pro molekulárné dynamické simulace s konstatntní teplotou. Pro Hamiltonian zavedeme jeden stupněm volnosti 
navíc pro tepelnou lázeň $s$.

% $$
% H(\vec p, \vec r, p_s, s) = \sum 
% $$

Pohybové rovnice pro jeden stupěň volnosti

\begin{align}
    \frac{dr}{dt} & = \frac{p}{m} \\
    \frac{dp}{dt} & = F -\frac{}{Q} p
\end{align}

\subsection{Langevin Termostat}

Stochastický termostat s disipací energie.

\subsection{Andersen Termostat}
Extrémní stochastický termostat kdy v každém simulačním kroku převzorkujeme momenty.


